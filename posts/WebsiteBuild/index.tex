% Options for packages loaded elsewhere
\PassOptionsToPackage{unicode}{hyperref}
\PassOptionsToPackage{hyphens}{url}
\PassOptionsToPackage{dvipsnames,svgnames,x11names}{xcolor}
%
\documentclass[
  letterpaper,
  DIV=11,
  numbers=noendperiod,
  oneside]{scrartcl}

\usepackage{amsmath,amssymb}
\usepackage{iftex}
\ifPDFTeX
  \usepackage[T1]{fontenc}
  \usepackage[utf8]{inputenc}
  \usepackage{textcomp} % provide euro and other symbols
\else % if luatex or xetex
  \usepackage{unicode-math}
  \defaultfontfeatures{Scale=MatchLowercase}
  \defaultfontfeatures[\rmfamily]{Ligatures=TeX,Scale=1}
\fi
\usepackage{lmodern}
\ifPDFTeX\else  
    % xetex/luatex font selection
\fi
% Use upquote if available, for straight quotes in verbatim environments
\IfFileExists{upquote.sty}{\usepackage{upquote}}{}
\IfFileExists{microtype.sty}{% use microtype if available
  \usepackage[]{microtype}
  \UseMicrotypeSet[protrusion]{basicmath} % disable protrusion for tt fonts
}{}
\makeatletter
\@ifundefined{KOMAClassName}{% if non-KOMA class
  \IfFileExists{parskip.sty}{%
    \usepackage{parskip}
  }{% else
    \setlength{\parindent}{0pt}
    \setlength{\parskip}{6pt plus 2pt minus 1pt}}
}{% if KOMA class
  \KOMAoptions{parskip=half}}
\makeatother
\usepackage{xcolor}
\usepackage[left=1in,marginparwidth=2.0666666666667in,textwidth=4.1333333333333in,marginparsep=0.3in]{geometry}
\ifLuaTeX
  \usepackage{luacolor}
  \usepackage[soul]{lua-ul}
\else
  \usepackage{soul}
\fi
\setlength{\emergencystretch}{3em} % prevent overfull lines
\setcounter{secnumdepth}{-\maxdimen} % remove section numbering
% Make \paragraph and \subparagraph free-standing
\ifx\paragraph\undefined\else
  \let\oldparagraph\paragraph
  \renewcommand{\paragraph}[1]{\oldparagraph{#1}\mbox{}}
\fi
\ifx\subparagraph\undefined\else
  \let\oldsubparagraph\subparagraph
  \renewcommand{\subparagraph}[1]{\oldsubparagraph{#1}\mbox{}}
\fi


\providecommand{\tightlist}{%
  \setlength{\itemsep}{0pt}\setlength{\parskip}{0pt}}\usepackage{longtable,booktabs,array}
\usepackage{calc} % for calculating minipage widths
% Correct order of tables after \paragraph or \subparagraph
\usepackage{etoolbox}
\makeatletter
\patchcmd\longtable{\par}{\if@noskipsec\mbox{}\fi\par}{}{}
\makeatother
% Allow footnotes in longtable head/foot
\IfFileExists{footnotehyper.sty}{\usepackage{footnotehyper}}{\usepackage{footnote}}
\makesavenoteenv{longtable}
\usepackage{graphicx}
\makeatletter
\def\maxwidth{\ifdim\Gin@nat@width>\linewidth\linewidth\else\Gin@nat@width\fi}
\def\maxheight{\ifdim\Gin@nat@height>\textheight\textheight\else\Gin@nat@height\fi}
\makeatother
% Scale images if necessary, so that they will not overflow the page
% margins by default, and it is still possible to overwrite the defaults
% using explicit options in \includegraphics[width, height, ...]{}
\setkeys{Gin}{width=\maxwidth,height=\maxheight,keepaspectratio}
% Set default figure placement to htbp
\makeatletter
\def\fps@figure{htbp}
\makeatother

\KOMAoption{captions}{tableheading}
\makeatletter
\makeatother
\makeatletter
\makeatother
\makeatletter
\@ifpackageloaded{caption}{}{\usepackage{caption}}
\AtBeginDocument{%
\ifdefined\contentsname
  \renewcommand*\contentsname{Table of contents}
\else
  \newcommand\contentsname{Table of contents}
\fi
\ifdefined\listfigurename
  \renewcommand*\listfigurename{List of Figures}
\else
  \newcommand\listfigurename{List of Figures}
\fi
\ifdefined\listtablename
  \renewcommand*\listtablename{List of Tables}
\else
  \newcommand\listtablename{List of Tables}
\fi
\ifdefined\figurename
  \renewcommand*\figurename{Figure}
\else
  \newcommand\figurename{Figure}
\fi
\ifdefined\tablename
  \renewcommand*\tablename{Table}
\else
  \newcommand\tablename{Table}
\fi
}
\@ifpackageloaded{float}{}{\usepackage{float}}
\floatstyle{ruled}
\@ifundefined{c@chapter}{\newfloat{codelisting}{h}{lop}}{\newfloat{codelisting}{h}{lop}[chapter]}
\floatname{codelisting}{Listing}
\newcommand*\listoflistings{\listof{codelisting}{List of Listings}}
\makeatother
\makeatletter
\@ifpackageloaded{caption}{}{\usepackage{caption}}
\@ifpackageloaded{subcaption}{}{\usepackage{subcaption}}
\makeatother
\makeatletter
\@ifpackageloaded{sidenotes}{}{\usepackage{sidenotes}}
\@ifpackageloaded{marginnote}{}{\usepackage{marginnote}}
\makeatother
\makeatletter
\makeatother
\ifLuaTeX
  \usepackage{selnolig}  % disable illegal ligatures
\fi
\IfFileExists{bookmark.sty}{\usepackage{bookmark}}{\usepackage{hyperref}}
\IfFileExists{xurl.sty}{\usepackage{xurl}}{} % add URL line breaks if available
\urlstyle{same} % disable monospaced font for URLs
\hypersetup{
  pdftitle={How I built my site},
  pdfauthor={Kate Hayes},
  colorlinks=true,
  linkcolor={blue},
  filecolor={Maroon},
  citecolor={Blue},
  urlcolor={Blue},
  pdfcreator={LaTeX via pandoc}}

\title{How I built my site}
\usepackage{etoolbox}
\makeatletter
\providecommand{\subtitle}[1]{% add subtitle to \maketitle
  \apptocmd{\@title}{\par {\large #1 \par}}{}{}
}
\makeatother
\subtitle{Migrating from blogdown to Quarto}
\author{Kate Hayes}
\date{2024-04-15}

\begin{document}
\maketitle
There's tons of resources out there for how to built a personal website
as an academic, made by folks with actual expertise in web design,
coding, etc.

I'd wanted to start my own blog /
\href{https://maggieappleton.com/garden-history}{digital garden} for
years. Initially, the roadblock was imposter syndrome - did I actually
have anything to write about? Would it be useful to anyone? isn't it all
a bit cringe?\sidenote{\footnotesize Counterpoints:

  \begin{enumerate}
  \def\labelenumi{\arabic{enumi}.}
  \tightlist
  \item
    I've never suffered from a lack of opinions
  \item
    writing is useful to me so it doesn't really matter if it's useful
    to anyone else
  \item
    cringe isn't real
  \end{enumerate}}

Once those worries were \st{shouted down} overcome, the main roadblock
became technical know-how. I originally developed my website using a
bundle of tools including blogdown, github, netlify all centered around
the very popular academic theme from hugo. I've had it for years now,
but it was heavy-lifting - building it took what felt like a week of
work back in 2020, and I couldn't update easily without having to
relearn the folder structure rules each time.

Originally, I wanted to tweak the theme, add some blog pages and be done
with it, but writeups like
\href{https://masalmon.eu/2020/02/29/hugo-maintenance/}{this} from
Maelle Salmon discussing hugo theme maintenance and this one from
Arielle on Quarto made me wonder if there was a better way.

The straw that broke the camel's back was \st{vanity} the
\href{https://quarto-dev.github.io/quarto-gallery/page-layout/tufte.html}{Tufte
theme} that's possible in Quarto - per the requirements of receiving a
degree in geography, my copy of \emph{Visual Display of Quantitative
Information} lives on my coffee table. Adding Tufte's beautiful
sidenotes and column spacing is as easy as adding a line or two of yaml
in Quarto and opens up a whole new world of snarky asides\sidenote{\footnotesize See:
  this entire writeup}.

\hypertarget{personal-website-development-101}{%
\subsubsection{Personal Website Development
101}\label{personal-website-development-101}}

Some asides before we get into the specifics:

``Have a personal website'' is advice you run into over and over when
thinking about a scientific career. Unfortunately, often that
conversation starts and stops there. Questions like how? where? what
goes on a personal website?

\hypertarget{switching-from-blogdown-to-quarto}{%
\subsection{Switching from blogdown to
quarto}\label{switching-from-blogdown-to-quarto}}

\hypertarget{initiating-a-quarto-website}{%
\subsubsection{Initiating a quarto
website}\label{initiating-a-quarto-website}}

Initiating a Quarto website is shockingly easy. Working from Rstudio,
you simply open a \textbf{New Project}, select \textbf{Quarto Website},
name it something new and click \textbf{Render} to preview.

That's it. 4 steps, X files, and you're previewing the structure. Now,
your render won't be hosted anywhere but locally and you haven't
attached it to Github yet, but already that's a much more easy entry
than the one I remember from Hugo.

\hypertarget{section}{%
\subsubsection{}\label{section}}

To add a blog:

initiatiate a quarto file for your landing page

add to menu bar

start a folder called posts

start to fill it with posts

\hypertarget{switching-netlify-hosts}{%
\subsubsection{Switching netlify hosts}\label{switching-netlify-hosts}}

I'd expected that my challenge would be deciphering the CSS / hugo
language, but really it ended up being deciphering the folder structure
within the website. Hugo hosts websites as static, meaning they compile
upon request (more detail on the distinction between static and dynamic
websites here)

\emph{Static vs Dynamic}

A static website delivers web pages exactly as they are stored (in this
case, as a series of folders and markdown files within a repository).
Dynamic websites generate content in real time, meaning they can deliver
web pages differently to different audiences (think a forum, in which
the forum page will appear differently depending on the user who's
logged in).



\end{document}
