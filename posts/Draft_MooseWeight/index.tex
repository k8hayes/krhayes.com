% Options for packages loaded elsewhere
\PassOptionsToPackage{unicode}{hyperref}
\PassOptionsToPackage{hyphens}{url}
\PassOptionsToPackage{dvipsnames,svgnames,x11names}{xcolor}
%
\documentclass[
  letterpaper,
  DIV=11,
  numbers=noendperiod,
  oneside]{scrartcl}

\usepackage{amsmath,amssymb}
\usepackage{iftex}
\ifPDFTeX
  \usepackage[T1]{fontenc}
  \usepackage[utf8]{inputenc}
  \usepackage{textcomp} % provide euro and other symbols
\else % if luatex or xetex
  \usepackage{unicode-math}
  \defaultfontfeatures{Scale=MatchLowercase}
  \defaultfontfeatures[\rmfamily]{Ligatures=TeX,Scale=1}
\fi
\usepackage{lmodern}
\ifPDFTeX\else  
    % xetex/luatex font selection
\fi
% Use upquote if available, for straight quotes in verbatim environments
\IfFileExists{upquote.sty}{\usepackage{upquote}}{}
\IfFileExists{microtype.sty}{% use microtype if available
  \usepackage[]{microtype}
  \UseMicrotypeSet[protrusion]{basicmath} % disable protrusion for tt fonts
}{}
\makeatletter
\@ifundefined{KOMAClassName}{% if non-KOMA class
  \IfFileExists{parskip.sty}{%
    \usepackage{parskip}
  }{% else
    \setlength{\parindent}{0pt}
    \setlength{\parskip}{6pt plus 2pt minus 1pt}}
}{% if KOMA class
  \KOMAoptions{parskip=half}}
\makeatother
\usepackage{xcolor}
\usepackage[left=1in,marginparwidth=2.0666666666667in,textwidth=4.1333333333333in,marginparsep=0.3in]{geometry}
\setlength{\emergencystretch}{3em} % prevent overfull lines
\setcounter{secnumdepth}{-\maxdimen} % remove section numbering
% Make \paragraph and \subparagraph free-standing
\ifx\paragraph\undefined\else
  \let\oldparagraph\paragraph
  \renewcommand{\paragraph}[1]{\oldparagraph{#1}\mbox{}}
\fi
\ifx\subparagraph\undefined\else
  \let\oldsubparagraph\subparagraph
  \renewcommand{\subparagraph}[1]{\oldsubparagraph{#1}\mbox{}}
\fi


\providecommand{\tightlist}{%
  \setlength{\itemsep}{0pt}\setlength{\parskip}{0pt}}\usepackage{longtable,booktabs,array}
\usepackage{calc} % for calculating minipage widths
% Correct order of tables after \paragraph or \subparagraph
\usepackage{etoolbox}
\makeatletter
\patchcmd\longtable{\par}{\if@noskipsec\mbox{}\fi\par}{}{}
\makeatother
% Allow footnotes in longtable head/foot
\IfFileExists{footnotehyper.sty}{\usepackage{footnotehyper}}{\usepackage{footnote}}
\makesavenoteenv{longtable}
\usepackage{graphicx}
\makeatletter
\def\maxwidth{\ifdim\Gin@nat@width>\linewidth\linewidth\else\Gin@nat@width\fi}
\def\maxheight{\ifdim\Gin@nat@height>\textheight\textheight\else\Gin@nat@height\fi}
\makeatother
% Scale images if necessary, so that they will not overflow the page
% margins by default, and it is still possible to overwrite the defaults
% using explicit options in \includegraphics[width, height, ...]{}
\setkeys{Gin}{width=\maxwidth,height=\maxheight,keepaspectratio}
% Set default figure placement to htbp
\makeatletter
\def\fps@figure{htbp}
\makeatother
\newlength{\cslhangindent}
\setlength{\cslhangindent}{1.5em}
\newlength{\csllabelwidth}
\setlength{\csllabelwidth}{3em}
\newlength{\cslentryspacingunit} % times entry-spacing
\setlength{\cslentryspacingunit}{\parskip}
\newenvironment{CSLReferences}[2] % #1 hanging-ident, #2 entry spacing
 {% don't indent paragraphs
  \setlength{\parindent}{0pt}
  % turn on hanging indent if param 1 is 1
  \ifodd #1
  \let\oldpar\par
  \def\par{\hangindent=\cslhangindent\oldpar}
  \fi
  % set entry spacing
  \setlength{\parskip}{#2\cslentryspacingunit}
 }%
 {}
\usepackage{calc}
\newcommand{\CSLBlock}[1]{#1\hfill\break}
\newcommand{\CSLLeftMargin}[1]{\parbox[t]{\csllabelwidth}{#1}}
\newcommand{\CSLRightInline}[1]{\parbox[t]{\linewidth - \csllabelwidth}{#1}\break}
\newcommand{\CSLIndent}[1]{\hspace{\cslhangindent}#1}

\KOMAoption{captions}{tableheading}
\makeatletter
\makeatother
\makeatletter
\makeatother
\makeatletter
\@ifpackageloaded{caption}{}{\usepackage{caption}}
\AtBeginDocument{%
\ifdefined\contentsname
  \renewcommand*\contentsname{Table of contents}
\else
  \newcommand\contentsname{Table of contents}
\fi
\ifdefined\listfigurename
  \renewcommand*\listfigurename{List of Figures}
\else
  \newcommand\listfigurename{List of Figures}
\fi
\ifdefined\listtablename
  \renewcommand*\listtablename{List of Tables}
\else
  \newcommand\listtablename{List of Tables}
\fi
\ifdefined\figurename
  \renewcommand*\figurename{Figure}
\else
  \newcommand\figurename{Figure}
\fi
\ifdefined\tablename
  \renewcommand*\tablename{Table}
\else
  \newcommand\tablename{Table}
\fi
}
\@ifpackageloaded{float}{}{\usepackage{float}}
\floatstyle{ruled}
\@ifundefined{c@chapter}{\newfloat{codelisting}{h}{lop}}{\newfloat{codelisting}{h}{lop}[chapter]}
\floatname{codelisting}{Listing}
\newcommand*\listoflistings{\listof{codelisting}{List of Listings}}
\makeatother
\makeatletter
\@ifpackageloaded{caption}{}{\usepackage{caption}}
\@ifpackageloaded{subcaption}{}{\usepackage{subcaption}}
\makeatother
\makeatletter
\@ifpackageloaded{sidenotes}{}{\usepackage{sidenotes}}
\@ifpackageloaded{marginnote}{}{\usepackage{marginnote}}
\makeatother
\makeatletter
\makeatother
\ifLuaTeX
  \usepackage{selnolig}  % disable illegal ligatures
\fi
\IfFileExists{bookmark.sty}{\usepackage{bookmark}}{\usepackage{hyperref}}
\IfFileExists{xurl.sty}{\usepackage{xurl}}{} % add URL line breaks if available
\urlstyle{same} % disable monospaced font for URLs
\hypersetup{
  pdftitle={How much does a moose weigh?},
  pdfauthor={Kate Hayes},
  colorlinks=true,
  linkcolor={blue},
  filecolor={Maroon},
  citecolor={Blue},
  urlcolor={Blue},
  pdfcreator={LaTeX via pandoc}}

\title{How much does a moose weigh?}
\author{Kate Hayes}
\date{2024-04-12}

\begin{document}
\maketitle
\hypertarget{how-do-you-determine-the-average-weight-of-a-moose}{%
\subsection{How do you determine the average weight of a
moose?}\label{how-do-you-determine-the-average-weight-of-a-moose}}

\emph{AKA - a field ecologist discovers how heartbreaking it is to bury
the enormity of work required to parameterize and benchmark a model into
the supplement section of a paper}

As part of my
\href{https://www.nsf.gov/awardsearch/showAward?AWD_ID=2219248&HistoricalAwards=false}{NSF
postdoctoral fellowship} exploring interactions between fire and biotic
disturbances in Interior Alaska, I've spent a lot of my time in the last
year parameterizing moose and hare for the model iLand using the biotic
disturbance model BITE. This has lead to some interesting conversations
over the last few months - several unsuspecting new introductions at
conferences have stared at me, slightly blankly, before asking ``so,
there are tiny moose in your model?''\sidenote{\footnotesize I would love}

BITE (full name: BIotic disTurbance Engine) is a module built by the
very smart Juha Honkankiemi (LUKE) and Werner Rammer (Technical
University of Munich) to simulate a whole world of biotic disturbances
(Honkaniemi, Rammer, and Seidl
2021)\marginpar{\begin{footnotesize}\leavevmode\vadjust pre{\protect\hypertarget{ref-honkaniemi2021}{}}%
Honkaniemi, Juha, Werner Rammer, and Rupert Seidl. 2021. {``From Mycelia
to Mastodons {\textendash} A General Approach for Simulating Biotic
Disturbances in Forest Ecosystems.''} \emph{Environmental Modelling \&
Software} 138 (April): 104977.
\url{https://doi.org/10.1016/j.envsoft.2021.104977}.\vspace{2mm}\par\end{footnotesize}}.
The term ``biotic disturbances'' is itself, a catchall - megafauna like
moose or mastodons are biotic disturbances, insects like aspen leaf
miner or emerald ash borer are biotic disturbances, fungi like running
canker or XX are biotic disturbances. Called ``agents'' for short, each
of these types of biotic disturbances occur in very different numbers
across very different scales - if we assume that mastodons behaved
similarly to modern elephants, they might have traveled up to 35 miles
within a day, while emerald ash borer are creeping across the United
States at a pace of a half a mile a year (assuming they don't get into
your firewood).

While a landscape model might be able to easily represent an individual
mastodon roaming across a landscape, modeling each individual fungi
spore across a landscape is absolutely not possible computationally. To
deal with this, BITE represents agents in units of biomass. Kilograms of
mastodon move across cells within the model, uprooting kilograms of
trees and consuming kilograms of spruce needles before falling into
kilograms of tar pits\sidenote{\footnotesize (technically no, though I'm sure if you
  thought too hard about it, you could add a raster of tar pits into the
  model)}.

This means you have two key decisions to make when starting to
parameterize a new agent:

\begin{enumerate}
\def\labelenumi{\arabic{enumi}.}
\item
  how many of them are they?
\item
  what do they weigh?
\end{enumerate}

Like so many things in science, answering those questions is way more
complicated than I'd expected.

\hypertarget{how-many-moose-are-there}{%
\subsubsection{How many moose are
there?}\label{how-many-moose-are-there}}

Because moose are so incredibly important for Alaska\sidenote{\footnotesize Moose are
  a critical subsistence resource for Alaskan communities, they regulate
  vegetation communities and they increase tips for tour bus drivers
  when spotted on a tour (citation - Brian Buma)}, we have a
surprisingly good grasp on their numbers - hunting keeps populations
steady at about XX moose per ha in different regions. Hunters are
required to provide harvest data, and while that data certainly has some
limitations, it does give us a good sense year to year for the density
of moose across the state.

Managers also monitor moose populations using something called the
\href{https://www.nps.gov/articles/000/moose-abundance-estimates.htm\#:~:text=In\%20Alaska\%20and\%20Canada\%2C\%20moose,geospatial\%20population\%20estimator\%20(GSPE).}{geospatial
population estimator protocol (GSPE)}. Using data from aerial count
surveys, managers use GSPE and other modeling approaches to estimate
populations across

See (Reinking et al.
2022)\marginpar{\begin{footnotesize}\leavevmode\vadjust pre{\protect\hypertarget{ref-reinking2022}{}}%
Reinking, Adele K., Stine Højlund Pedersen, Kelly Elder, Natalie T.
Boelman, Thomas W. Glass, Brendan A. Oates, Scott Bergen, et al. 2022.
{``Collaborative Wildlife{\textendash}snow Science: Integrating Wildlife
and Snow Expertise to Improve Research and Management.''}
\emph{Ecosphere} 13 (6): e4094. \url{https://doi.org/10.1002/ecs2.4094}.\vspace{2mm}\par\end{footnotesize}}
for more on the important relationship between wildlife and snow and
also the incredible phrasing of ``snow professionals'' and ``snow
science community''\sidenote{\footnotesize I love my job}.

All to say, we can say with some (some!) certainty that there are
somewhere between XX and XX moose in Interior Alaska, about XX per acre.

\hypertarget{what-do-moose-weigh}{%
\subsubsection{What do moose weigh?}\label{what-do-moose-weigh}}

Well, I started with the obvious route. Googling ``how much does a moose
weigh'' doesn't get you very far. Plus, moose body mass differs across
the regions where they live - moose are generally largest in Alaska, and
smaller further south (Pastor et al.
1988)\marginpar{\begin{footnotesize}\leavevmode\vadjust pre{\protect\hypertarget{ref-pastor1988}{}}%
Pastor, John, Robert J. Naiman, Bradley Dewey, and Pamela McInnes. 1988.
{``Moose, Microbes, and the Boreal Forest.''} \emph{BioScience} 38 (11):
770--77. \url{https://doi.org/10.2307/1310786}.\vspace{2mm}\par\end{footnotesize}}.
Googling ``how much does a moose weigh alaska'' leads to a page from the
Alaska Department of Fish and Game that gives moose an average weight of
550 kilograms. They cite Bishop 1988, who puts bull moose between 540 -
680 kilograms and cows between 360 - 590 kg in Alaska.

\hypertarget{body-mass-differs-by-gender}{%
\subsubsection{Body Mass differs by
Gender}\label{body-mass-differs-by-gender}}

{[}{]} puts the weight of an average moose at XX kilograms. {[}{]}
points out that body mass is different between male and female moose -
{[}{]} points out that body mass of female moose differs according to
the number of calves they've produced. The number of calves they produce
depends in turn on the conditions that year - was there enough food? was
it warm enough?

So, if moose weight depends on the gender ratio across the population,
what's the gender ratio of the moose population in Alaska? Surprisingly,
we have a good grasp of this - hunting targets male moose, so
populations are maintained at a level of Xx moose per in. Fish and Game
reports the densities per game unit. The model landscape I use for this
work falls within game unit 20A, and a 2014 report shows a ratio of




\end{document}
